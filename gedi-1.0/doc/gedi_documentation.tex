\documentclass[10pt,a4paper,titlepage]{article}

\parskip=10pt
\parindent=0pt

\title{\bf The GeDI Document}
\author{Makia Minich (makia@llnl.gov)}
\date{\today}

\usepackage{multirow}
\usepackage{color}
\usepackage{fullpage}
\usepackage{fancyhdr}

\pagestyle{fancy}
\fancyhf{}
\fancyfoot[R]{\thepage}
\fancyfoot[L]{\bfseries GeDI Documentation}
\renewcommand{\headrulewidth}{0pt}
\renewcommand{\footrulewidth}{1pt}

\makeatletter
\def\maketitle
{
  \null
  \thispagestyle{empty}
  \vfill
  \begin{center}\leavevmode
    \normalfont
    {\LARGE\raggedleft \@author\par}
    \hrulefill\par
    {\huge\raggedright \@title\par}
  \end{center}
  \vfill
  {\Large \@date\par}
  \null
  \cleardoublepage
}
\makeatother

\begin{document}
\maketitle
\pagenumbering{roman}
\tableofcontents
\clearpage
\pagenumbering{arabic}
\section{Information}
GeDI (the GEneric Diskless Installer) is a tool to help aid in the setup of
diskless Linux clients.  Made up of a set of shell scripts (most of which
written in bash) the goal is to provide an easy way to configure and manage
one or greater Linux systems with minimal administrative overhead.  This
document is expecting at least some knowledge of the idea of network booting
a system as well as a base knowledge of Linux systems administration.  Because
each system can handle network booting differently,
Google\footnote{http://www.google.com} is a good place to look if you need more
information on how this works.

We'd recommend reading through this entire document before you start working on
your image.  Primarily because we took the time to write it all, but also
because it might answer questions that you didn't even know you had.

\section{The Hardware Setup}
GeDI is specifically written for a management/client model.  Namely, there is a
specified management server whose sole duty is to run the clients, and a set of
clients who allow for interaction from users.  Depending on the number of
clients you are looking to implement, your management node should be able to
handle them accordingly.  Therefore, the biggest components to focus on are
network speed, memory size and processor speeds.  In addition, a large enough
hard drive to hold the management node's root filesystem as well as the clients'
root filesystem is a must if you plan on using an NFS export from the management
node.  In an ideal setup, your mangement system should have some redundancy
options available, such as RAID-1 (for mirroring).  Not much is required from
the clients perspective.  They must have the ability to network boot in some
fashion (e.g. PXE-boot), and it is usually a good idea to have a fair amount of
memory available for user interaction, keeping in mind that the root
filesystem may need to use some of this memory.

Table \ref{tab:sysConfig} shows the setup that we're using (which will be the
example throughout this document).  Not much is really needed from the hardware
standpoint.  The most important component for the entire setup is a stable and
reliable network.  Without this the clients will have a hard time talking to
the management node, and could render them completely useless.

\begin{table}[ht]
\center
\caption{System Layout}
\begin{tabular}{|r||l|}
\hline
\multirow{4}{*}{Management Node} & 4G Ram \\
& Dual Processor Xeon 2.8Ghz (hyperthreading enabled) \\
& Gigabit Ethernet \\
& 4 40G SCSI Hard Disk Drives (Hardware RAID-1 in two pairs) \\
\hline\hline
\multirow{4}{*}{Clients (1 or more)} & 1G Ram \\
& Single Processor Pentium 4 2.8GHz \\
& 10/100baseT Ethernet \\
& no hard disk drive \\
\hline\hline
Network Equipment & 10/100baseT Switch with GigE uplink \\
\hline
Serial Console & Used to manage the management node \\
\hline
\end{tabular}
\label{tab:sysConfig}
\end{table}

\section{The Software Setup}
GeDI consists of one server piece, the GeDI rpm, and a client piece.  The server
peice contains all of the tools needed to create an image for a diskless client
and configure it for use.  The client piece (which should be installed in the
image) contains scripts to modify the image into something useable after boot.
In addition, you will need a few other linux tools to succesfully network boot
and load your client nodes.  Included are:
\begin{itemize}
\parskip=0pt
\item dhcp-2.0pl5-8 or any better dhcp server
\item tftp-server-0.28-1 or any better tftp server
\item xinetd for the tftp-server
\item nfs-utils for nfs exports
\item rpm-4.0.2-8 or anything higher
\end{itemize}

Most of these are relatively standard issue for an RPM based linux distribution,
and should be readily available.  In addition, your management node should have
a repository of all of the RPMS that you wish to install onto your diskless
image.  GeDI, by default, will look in \verb!/usr/gedi/RPMS! for these RPMS (in
fact, the first RPM one should put in this directory is gedi-client).

\section{The Directory Structure of GeDI}
When the GeDI RPM is installed, a number of directories and files get created on
the management node.  Please see table \ref{tab:dir} for the list.

\begin{table}[ht]
\centering
\caption{GeDI Directory Structure on the Management Node}
\begin{tabular}{rp{4in}}
\verb!/etc/gedi! & Directory containing the MAC.info file, rpmlists and variables files. \\
\verb!/usr/gedi/local! & Directory where the client's local customizations go. \\
\verb!/usr/gedi/nfsroot! & Directory where the client's image will be created. \\
\verb!/usr/gedi/patches! & Patches that should be applied to the installing image. \\
\verb!/usr/gedi/pxelinux.0! & Boot file sent to the client aiding in PXE-booting. \\
\verb!/usr/gedi/RPMS! & A repository of RPMS that are available to be installed on the client system. \\
\verb!/usr/gedi/scripts! & Scripts used to create the images. \\
\verb!/usr/gedi/tools! & Scripts made available as tools to help manage the system. \\
\end{tabular}
\label{tab:dir}
\end{table}

For you, the most important of these directories are etc, RPMS, scripts and
tools.  These are the places where you'll do most of your work in configuring
the system.  One important thing to note here is that the gedi RPM is actually
relocatable.  So, if you don't happen to like \verb!/usr/gedi! (or you want to
use a bigger disk mounted in some other location) then you can pass the
"\verb!--prefix!" option to RPM on the install line.  For example:

\begin{verbatim}
# rpm -ivh --prefix /tftpboot gedi-0.4.0-1.noarch.rpm
\end{verbatim}

If you don't want to relocate the RPM install, another option is available as
well.  Each location of the directories is controlled by a variable in the
\verb!/etc/gedi/variables! file.  So if you, perhaps, wanted to place your
RPMS in /usr/local/RPMS, you can change the associated variable in the
\verb!variables! file.

On the client side, you'll want to install the \verb!gedi-client! RPM.  See
table \ref{tab:clidir} for the list of directories.

\begin{table}[ht]
\centering
\caption{GeDI Directory Structure in the Client Image}
\begin{tabular}{rp{4in}}
\verb!/etc/gedi! & Directory containing the \verb!gedi_client.conf! configuartion file. \\
\verb!/sbin/mkinitrd_gedi! & Script to create the ramdisk for GeDI. \\
\verb!/usr/share/gedi/scripts! & Support scripts to get the image running. \\
\verb!/usr/share/gedi/src! & UnionFS source files. \\
\end{tabular}
\label{tab:clidir}
\end{table}

\section{A Note About Client Types}
GeDI has been created with the idea that you could be serving different types of
clients from a single management server.  For example, if you have some clients
where people would be compiling a lot of code, and other clients where this is
restricted you could easily create two node types--e.g. \verb!compile!,
\verb!not_compile!.  Every script that works on an image will require you to
supply this node type on the command line, so you might want to think about what
you'd like to use before jumping into things.  In the next section, we'll
actually see how this node type is used.

\section{Creating Your First Image}

\subsection{The RPM List}
GeDI is based on an NFSroot style setup of a diskless client.  In short this
means that, for each of the clients, the root filesystem will be made
available over NFS instead of found locally on the hard drive.  For this to
work, we'll need an NFS exported directory from the management node that
contains all of the elements of a working system.  This can be accomplished
through the use of an RPM list, and the "\verb!create_nfsroot!" script.  The
RPM list is made up of all of the RPMS that you wish to install on your system,
and looks something like:

\begin{verbatim}
Canna-libs-3.5b2-62.i386.rpm
ElectricFence-2.2.2-8.i386.rpm
Glide3-20010520-13
LPRng-3.8.9-4*
\end{verbatim}

Each line is made up of one RPM, and is comprised of it's filename (GeDI can
handle file globbing and name expansion).  The actual sorting of the list
doesn't really matter (though it has been found that properly sorted lists tend
to give less errors when they are being installed).  If you have a running
system that you'd like to base your client off of (which is the easiest thing
to do) you can run the following command on that system:

\begin{verbatim}
# rpm -qa --queryformat "%{NAME}-%{VERSION}-%{RELEASE}.%{ARCH}.rpm\n" > /tmp/rpmlist
\end{verbatim}

This will put (into the file \verb!/tmp/rpmlist!) the list of all of the RPMS
that you have on your running system.  This file can then be directly used by
GeDI to create your image.

Once you have the list all setup, you can put it into the \verb!/etc/gedi!
directory.  If you have determined a node type for this RPM list, you will want
to name the file \verb!rpmlist.<nodetype>! (e.g. \verb!rpmlist.other!).  The
"\verb!rpmlist!" portion of the name is unimportant, but it's a bit easier
to call it "\verb!rpmlist!" to know what it is.  On the otherhand, the
"\verb!.<nodetype>!" portion is quite important as it will tell the
\verb!create_nfsroot! script that you are looking to create an image for a
specific nodetype (\verb!<nodetype>!), so be careful of what you call it.

\begin{quote}
\bf**NOTE**\par
This is a pretty good time to mention a little bit about the client's kernel.
The latest version of GeDI makes use of an initial ramdisk to allow you to boot
any available kernel on your clients.  Put slightly different, we no longer
require a special re-built kernel for the client to boot up.  When you install
a new kernel into the image, you'll want to run
\verb!/usr/gedi/scripts/create_ramdisk! (which is automatically run from the
\verb!configure! script).  This will go inside of the image, and create a
ramdisk for each kernel you have available.  This ramdisk will contain any
useful network drivers and have the ability to NFS mount your root filesystem.
In addition, the script will create the \verb!gedi_pxe! and \verb!boot.msg!
files in the images \verb!/boot! directory which are used to allow you a way
to select which kernel to boot on while the system is coming up (much like
grub or lilo on diskfull systems).
********
\end{quote}

Before you can count your RPM list as finished, make sure that you add the
\verb!gedi-client! RPM.  This contains useful scripts that will get your system
booted and running (including the ability to create ramdisks).

\subsection{Test Your RPM List}
Once you have your RPM list all set up, it's time to test your list.  This step
is not required, but highly recommended since it might save some time in the
long run.  To test your RPM list, you can run the following command:

\begin{verbatim}
# /usr/gedi/scripts/create_nfsroot -t <RPM List>
\end{verbatim}

If you happened to put your rpmlist in \verb!/etc/gedi!, you can substitue
\verb!<RPM List>! with just the name of the file, otherwise you'll have to give
the full path to the file.  Remember, though, that the filename must end with
"\verb!.<nodetype>!" or things might get a bit confusing.  After a little bit
of time, you should see what appears to be your RPM's installing.  Actually,
what is happening is this:
\begin{enumerate}
\parskip=0pt
\item We check your RPM list to make sure that all of the files exist.
\item We perform a chroot'ed RPM-database install of the selected RPM's.
\item We remove our test directory.
\end{enumerate}
This database install is a simple way to check if all of the dependencies for
your RPM list are met.  No actual RPM's are installed but the test RPM database
is told that they are.

\subsection{Customize Your Image}
To aid in the local customization of your image, we have created the
\verb!/usr/gedi/local! directory.  This directory should contain copies of all
of the files that you would like to have installed in your image as well as a
configuration file (\verb!local.conf!) to detail their permissions and
destination.  The \verb!local.conf! file has the following layout:

\begin{verbatim}
<local filename>    <image>   <destination>      <mode>    <user>    <group>
\end{verbatim}

The \verb!<local filename>! is the name of the file in \verb!/usr/gedi/local!.  
\verb!<image>! is the name of the image that the file should be installed into
(using "*" will install the file into all of the images).  \verb!<destination>!
is the full path and resulting filename where it should be copied to.
\verb!<mode>! is the mode that will be passed to chmod for setting the correct
permissions on the file.  \verb!<user>! and \verb!<group>! are used to set
ownership on the file.  For example, to copy \verb!hosts.allow! to the compute
image as \verb!/etc/hosts.allow! with owner and group of root and permissions of
644, we'd use the following line in \verb!local.conf!:

\begin{verbatim}
hosts.allow         compute   /etc/hosts.allow   644       root      root
\end{verbatim}

Or, to install the same file into all images, we'd use:

\begin{verbatim}
hosts.allow         *         /etc/hosts.allow   644       root      root
\end{verbatim}

This format allows us to keep a flat directory, but keep the final destination
for the file known.  If, perhaps, you instead wanted to use the management nodes
actual \verb!hosts.allow! file (which should be located in \verb!/etc!), you
could instead do the following:

\begin{verbatim}
/etc/hosts.allow    *         /etc/hosts.allow   644       root      root
\end{verbatim}

This tells GeDI to actually pull the real file from a different location than
the \verb!local! directory.

\subsection{Create Your Image}
Now, if we see that all of our dependencies are fulfilled and our
customizations are in place we are ready to actually create our image.  To do
this, we can run the following (notice the lack of a '-t'):
\begin{verbatim}
# /usr/gedi/scripts/create_nfsroot <RPM List>
\end{verbatim}
Now, you can probably step away for a bit since this usually takes a little
while (though we could possibly go on to the next steps and let this finish
in the background).

\section{Getting the Server Ready}
If all went well, we should now have an image to serve out, but we still need
to get everything ready on the managment node.  The first, and most important,
piece is the MAC.info file.  This is a file that lives in \verb!/etc/gedi!
and contains a mapping between a hostname, IP-Address, hardware MAC-Address,
and node type.  The file looks something like this:

\begin{verbatim}
00:00:00:00:00:00            hostname      192.168.1.1      other
\end{verbatim}

\verb!/usr/gedi/tools/MAC_Collector! is a simple script that helps aid in the
process of discovering clients.  Basically, it listens for broadcasts on your
ethernet network, and upon receiving one it asks questions about that node
(including whether or not the system found is the client you are expecting).
It then performs all of the necessary system configuration.  It's quite handy
when it works, but the script can be somewhat unreliable on networks that
experience heavy traffic.  Therefore, you might need to find a more reliable way
to collect MAC addresses.  If you would like to use a network-based method, you
can use the following command (which is the heart of the \verb!MAC_Collector!
script):

\begin{verbatim}
# /usr/sbin/tcpdump -i <Ethernet Device> -c 1 -entl port bootpc 2>/dev/null |
   /bin/awk '{
      split($1, sep, ":")
      printf("%02s:%02s:%02s:%02s:%02s:%02s\n", sep[1], sep[2], sep[3],
                                                sep[4], sep[5], sep[6])
   }'
\end{verbatim}

This will return one MAC address and display it on STDOUT.  Another method is to
actually watch the screen of the network booting system.  Generally, when the
PXE Boot message appears, along with it comes the MAC Address.  Depending on the
system, you might need to quickly write it down, as it might not stay on screen
for too long.

Another option is to just use ranges in the \verb!MAC.info! file.  This layout
looks something like:

\begin{verbatim}
192.168.1.1-192.168.1.10     other
\end{verbatim}

The first column contains the IP-Address range that you'd like to use for the
node type you defined in the second column.  This method has the benefit of
being able to boot clients without having to collect MAC addresses, but keep
in mind that it will then attempt to listen to any dhcp traffic on your network
and try to boot those requestors as well.

After you've got your \verb!MAC.info! file setup the way that you like it, you
can run the \verb!config_services! script in \verb!/usr/gedi/tools!, which will
perform the remaining actions needed to get your system all set up and ready.
The \verb!config_services! script is given as an aid in configuring all of the
services that are needed to get GeDI up and going.  By default, this script will
do the following:

\begin{enumerate}
\parskip=0pt
\item Create \verb!/etc/dhcpd.conf!, and populate it with the systems listed in MAC.info.
\item Add entries to \verb!/etc/exports! to NFS export the correct image to each system.
\item Add entries to \verb!/etc/hosts! for each system listed in MAC.info.
\item Restart dhcpd.
\item Re-export all of the NFS exports.
\end{enumerate}

Once all of this is done, your server should be ready to boot your clients.

\begin{quote}
\bf**NOTE**\par
The modification of the exports, hosts and other files can be toggled by
modifying the variables "\verb!MODIFY_EXPORTS!", "\verb!MODIFY_HOSTS_FILE!", and
any other "\verb!MODIFY_!" entries in the file \verb!/usr/gedi/etc/variables!.
By default these are turned on, so if you want to control what happens to your
exports and hosts make sure and set these to "NO".\par
********
\end{quote}

\section{Booting Your First Node}
At this time, you should be just about ready to go.  As a checklist before
booting your node, make sure that you have the following services started and
configured on the management node:

\begin{itemize}
\parskip=0pt
\item NFSD
\item dhcpd
\item tftpd (usually set through xinetd)
\end{itemize}

Without these services, you're clients will be dead in the water.  So, if
everything is running, you should be able to boot up a client and see the
following happening:

\begin{enumerate}
\parskip=0pt
\item The client powers up and starts going through it's BIOS messages.
\item The client should attempt to start network/PXE booting.
\item The server should receive a broadcast from the client and send some network configuration information to the client.
\item The client will then ask for a file to boot off of.
\item The server will send the pxelinux.0 file to the client as well as the name of the configuration file it should use to boot.
\item The client will load and run pxelinux.0, which basically tells the client to send a request for the appropriate configuration file.
\item The server will send out the related file.
\item The client will read the file and figure out the name of the kernel to ask for.
\item The server will send the correct kernel to the client.
\item The client will boot the received kernel with a specified kernel-command line and linux will start loading.
\item When the root filesystem is about to be mounted, the client will issue an NFS request to the server for the appropriate mount.
\item The client will mount the NFS-root filesystem read-only as well as a writeable RAM disk.
\item The NFS mount and RAM disk are layered into a unionfs filesystem which is used as the root filesystem.
\item The system pivot-roots to the new unionfs filesystem and starts booting the correct runlevel.
\end{enumerate}

If all went well, you should now have a working client to use and enjoy.

\section{Managing The Images After Everything Else Is Finished}
So, if you've done all of the steps correctly, you should have a set of working
diskless clients up and running.  Pretty cool, huh?  Now comes the time of how
to manage these systems.

\subsection{Adding Users}
First, we'll deal with adding users to the clients.  If you happen to standardly
use local passwords on systems (or some various form of pam-authentication type)
you can use the sync\_passwd script in \verb!/usr/gedi/tools!.  This simple
script will take the passwd (and associated) files from the management node and
install them into the images that you are hosting.  This can be run from cron
on the management node so that all you need to worry about doing is updating one
system's passwd entries.  If you are worried about restricting users from
accessing your management system, you might take a look at the file
\verb!/etc/security/access.conf! which is installed by pam.  With some tweaking,
you can restrict access to your management node, while allowing access to your
clients all with the same passwd files.

\subsection{Changing Configuration Files}
Now, let's say we need to change some configuration files that are being placed
into the images.  This is easily done by using the \verb!local! directory and
the \verb!configure! script.  Just place those pesky configuration
files into \verb!/usr/gedi/local!, modify the \verb!local.conf! file and re-run
\verb!/usr/gedi/scripts/configure! to put them in place.  If the client is
running, and this is a change that needs to somehow be put into affect (e.g.
restarting a network service) you might need to actually perform that action on
the clients.

Because GeDI is using unionfs, there's a chance that you'll find that your
clients may not see the change that you made on the management node.  Generally
this will happen if, at some point after booting, the specific file has been
modified locally on your client.  If this does happen, and you want to fix it,
you will need to reboot your system (or just install the new version locally
as well).

\subsection{Local Modifications - Image}
OK, that covers the handling of configuration files, but sometimes we also need
some actual filesystem manipulations to happen (e.g. make links to directories,
turning off and on various services).  This can be done with the special
\verb!local_mods! script.  \verb!/usr/gedi/scripts/local_mods! (which is run
everytime \verb!configure! is run) is a simple (or more complex if you'd like
to make it that way) script that will be run before copying files from the
local directory.  GeDI provides an example \verb!local_mods! script in the
\verb!/usr/share/doc/gedi-<version>/examples! directory so that you can take a
look at what kind of things can be done.  Remember that this script is written
from the point of view outside of the image, so you need to be especially
careful in what you are doing.  The example references the
\verb!/etc/gedi/variables! file, which can be helpful in not forcing you
to remember the directory structure.

\subsection{Local Modifications - Booting System}
While the \verb!local_mods! script modifies the system from the image
point-of-view, we should also talk about how to modify the image of a booting
system.  This can be done one of two ways:

\begin{itemize}
\item Create a \verb!rc.local! script and place it in the \verb!/usr/gedi/local!
directory.
\item Modify \verb!/usr/gedi/local/rc.readonly! with the changes that you want.
\end{itemize}

The \verb!rc.local! is probably the most generic way to handle cusomizations to
your system (such as modprobing specific modules and whatnot).  This way does
work, but remember that \verb!rc.local! is generally the very last script to be
run at boot time.  The \verb!rc.readonly! script is a RedHat'ism (introduced in
RHEL4) that is actually run from \verb!rc.sysinit! early in the boot process.
So, if you had a filesystem that you wanted to mount long before the services
start running, this is your best bet.

\subsection{Image Upgrades}
So, now what happens when we want to upgrade the image or even install new
packages?  Well, there are a few ways to handle this, it all depends on personal
preference.

\begin{enumerate}
\item {\bfseries Just do it by hand and damn the consequences.}

This way makes life in the short-term nice and easy.  If you have a new RPM to
install, just install it (or upgrade it) and be done with it.  This is easy
enough to accomplish with the following command:

\begin{description}
\item {To install:}
\begin{verbatim}# rpm -ivh -r <PATH TO IMAGE> <RPM TO INSTALL>\end{verbatim}
\item {To upgrade:}
\begin{verbatim}# rpm -Uvh -r <PATH TO IMAGE> <RPM TO INSTALL> \end{verbatim}
\item {To freshen:}
\begin{verbatim}# rpm -Fvh -r <PATH TO IMAGE> <RPM TO INSTALL> \end{verbatim}
\end{description}

Not too bad, everything should go as planned and the image should have the new
RPM.  The only problem is that you don't have this change documented in the
RPM list.  This brings us to the next idea.

\item {\bfseries Edit the associated RPM list and manually upgrade the image.}

Basically, we'll do everything from step 1, but make sure that we also make the
needed changes in the rpmlist that we use (just in case we need to create a new
image for whatever reason).  This can be quite handy in the long run and not
too difficult to follow through on.

\item {\bfseries Use some handy scripts to keep things in sync.}

GeDI comes with a few tools to aid in making sure that your image matches what
you think should be installed (at least RPM-wise).  The first is
\verb!update_rpmdir!.  If you have a central repository set up where you are
keeping RPM updates, this script can help you keep \verb!/usr/gedi/RPMS!
up-to-date as well.  As an example, you can run the following:

\begin{verbatim}
# /usr/gedi/tools/update_rpmdir -u <PATH TO UPDATE RPMS>
\end{verbatim}

This will take all of the RPMs found in your updates directory, and copy any
newer packages into \verb!/usr/gedi/RPMS! so that they are ready to be put into
the image.  Once this is done, we can now use the \verb!update_rpmlist! script
to get our rpmlist to match what's newly available.  For example, you can run
the following:

\begin{verbatim}
# /usr/gedi/tools/update_rpmlist <RPM List>
\end{verbatim}

This will compare what packages are available in the RPM directory to the items
in your list.  If there's anything newer available in the directory, your list
will be updated to this new version (any errors should be displayed on STDOUT
and probably need to be taken care of before moving on).  Now that we have an
updated list, we can see what changes need to be made to the actual image.  This
can be done with the \verb!rpmdiff! script.  This script compares what's in the
list to what is currently in the image's RPM database and reports any
discreprencies:

\begin{verbatim}
# /usr/gedi/tools/rpmdiff <RPM List> <Nodetype>
\end{verbatim}

The script also has the option of modifying your \verb!rpmlist! and \verb!image!
with the actual changes that need to happen.  This can be done with the
\verb!-g! option:

\begin{verbatim}
# /usr/gedi/tools/rpmdiff -g <RPM List> <Nodetype>
\end{verbatim}

Pretty simple, eh?  Well, this way is a bit more tedious, but ensures that your
RPM repository and your list and your image are all in sync, which long-term
can help rule out certain issues if something goes wrong.
\end{enumerate}

\subsection{Adding and Removing Clients}
Since it is highly unlikely that from the first time you install your system
you'll never change what clients are connected, we should probably talk a little
bit about how to add and remove clients from GeDI.  This isn't actually all that
tedious of a task (especially with later versions of GeDI).  First, we have to
remember that the following files have an affect on what clients get booted and
what they boot:

\begin{itemize}
\parskip=0pt
\item \verb!/usr/gedi/etc/MAC.info!
\item \verb!/etc/dhcpd.conf!
\item \verb!/etc/exports!
\item \verb!/etc/hosts!
\item \verb!<image>/boot/pxe/gedi_pxe!
\end{itemize}

By default, GeDI's \verb!config_services! script can handle the modifications
of each of the system files based on changes that you make to the
\verb!MAC.info! file.  So, as an example, if we wanted to change the MAC address
of one client, remove another client, add other client, change an IP address,
and also change the options for an existing client, this can all be done by
modifying the \verb!MAC.info! file, and re-running \verb!config_services!.  This
is much better than having to go and fix all of the files by hand.

\begin{quote}
\bf**NOTE**\par
Remember, the modification of most configuration of the files is dependent on
the "\verb!MODIFY_!" variables set in the \verb!/etc/gedi/variables! file.
If these are set to no, you'll need to modify those files by hand.\par
********
\end{quote}

\section{GeDI Architecture}
Not in the realm of how to make it work, this is more of a what's under the
covers section.  GeDI was created from the "single image shared among many
systems" point-of-view.  Therefore, GeDI needed a way to allow one image to be
created, but allow the client nodes the ability to modify the image to suit
their needs (e.g. sendmail really likes to modify files in /etc/mail).  Early
incarnations of GeDI use the read-only NFSRoot method utilizing a local ramdisk.
Each node would boot up on the read-only root filesystem, and early in the boot
process they would create a ramdisk, and start populating it with the files that
needed to be written to.  While this does work, there is a certain amount of
management overhead as you try to keep an eye on every server that needs to
modify a file and make sure that the file is available to be written.  In
addition, this function required that at image creation, certain files were
removed and turned into symlinks that pointed into the non-existant ramdisk.
This could potentially upset the RPM verification process (as files are modified
and removed).

A cleaner solution was found in using UnionFS\footnote{See
http://www.unionfs.org for more information.}.  This allowed us to take the
symlinks to ramdisks completely out of the equation.  Instead, what we do is
mount the root filesystem read-only (just as before) and create/mount a ramdisk.
Utilizing UnionFS, we then merge the two filesystems, layering the writeable
ramdisk on top of the read-only NFS mount.  The result is then used as the
root filesystem, and the system continues booting up.  Any files that are
modified are automagically copied from the NFS filesystem into the ramdisk and
become writeable, allowing for local copies to be used.

Now, there can be an issue with UnionFS that still needs to be addressed (but
the same issue also existed with the prior symlink-ramdisk idea).  What happens
if you want to modify a file in the image, but a client has a local copy that
it has written to?  There are a couple of ways to deal with this (at the
moment).  The first is to go ahead and reboot the node(s), on reboot the
modifications are lost and you're back to your original image.  The second is
to remove the local copy from the \verb!/union/writeable! directory structure,
but according to the UnionFS developers this is not a recommended practice.

With later releases of GeDI, you now have a choice as to which method (UnionFS
or RAM-based) to use on the client.  By default, GeDI will attempt to build
and install the UnionFS modules, but if they aren't available, GeDI will instead
boot using the RAM-based method (creating writeable copies of /etc and /var).

\section{Conclusion}
Surely there's more that needs to be written here, but at the moment this is
all that there is.

\section{Side Notes}
As people start using GeDI (which I hope they do), we'll start adding
information to this section on various findings that come up.

\subsection{SELinux}
SELinux support in GeDI is non-existant and you might see problems with your
system caused by SELinux.  For example, if you start seeing weird errors while
running create\_image, there's a good chance that SELinux is trying to ``help''
you out by not allowing you to write certain files.  The way to fix this is to
disable SELinux via the kernel command line (e.g. in \verb!grub.conf! add
\verb!selinux=0! to the kernel line for your kernel).  Also, if you happen to
see weird ``cannot write'' messages from your client (most noticeable is with
\verb!ssh!) you should also disable SELinux there.  To do this, add
\verb!selinux=0! to the \verb!BOOT_OPTIONS! line in the GeDI \verb!variables!
file (don't forget to run \verb!configure! when you're done).
\end{document}
